\documentclass[
]{jss}

\usepackage[utf8]{inputenc}

\providecommand{\tightlist}{%
  \setlength{\itemsep}{0pt}\setlength{\parskip}{0pt}}

\author{
Boyi Guo\\University of Alabama at Birmingham \And Nengjun
Yi\\University of Alabama at Birmingham
}
\title{The R Package \pkg{BHAM}: Fast and Scalable Bayeisan Hierarchical
Additive Model for High-dimensional Data}

\Plainauthor{Boyi Guo, Nengjun Yi}
\Plaintitle{The R Package BHAM: Fast and Scalable Bayeisan Hierarchical
Additive Model for High-dimensional Data}
\Shorttitle{\pkg{BHAM}: Bayeisan Hierarchical Additive Model}


\Abstract{
\textbackslash pkg\{BHAM\} is a freely avaible R pakcage that implments
Bayesian hierarchical additive models for high-dimensional clinical and
genomic data. The package includes functions that generlized additive
model, and Cox additive model with the spike-and-slab LASSO prior. These
functions implements scalable and stable algorithms to estimate
parameters. \textbackslash pkg\{BHAM\} also provides utility functions
to construct additive models in high dimensional settings, select
optimal models, summarize bi-level variable selection results, and
visualize nonlinear effects. The package can facilitate flexible
modeling of large-scale molecular data, i.e.~detecting succeptable
variables and inforing disease diagnostic and prognostic. In this
article, we describe the models, algorithms and related features
implemented in \textbackslash pkg\{BHAM\}. The package is freely
avaiable via the public GitHub repository
\url{https://github.com/boyiguo1/BHAM}.
}

\Keywords{additive model, spike-and-slab LASSO, scalable}
\Plainkeywords{additive model, spike-and-slab LASSO, scalable}

%% publication information
%% \Volume{50}
%% \Issue{9}
%% \Month{June}
%% \Year{2012}
%% \Submitdate{}
%% \Acceptdate{2012-06-04}

\Address{
    Boyi Guo\\
    University of Alabama at Birmingham\\
    1665 University Blvd\\
Birmingham, AL 35294-0002 USA\\
  E-mail: \email{boyiguo1@uab.edu}\\
  URL: \url{http://boyiguo1.github.io}\\~\\
      Nengjun Yi\\
    University of Alabama at Birmingham\\
    1665 University Blvd\\
Birmingham, AL 35294-0002 USA\\
  E-mail: \email{nyi@uab.edu}\\
  
  }

% Pandoc syntax highlighting

% Pandoc citation processing


\usepackage{amsmath}

\begin{document}

\section{Introduction}

High-dimensional statistics has been an indispensable area of research
for its high impact in molecular and clinical data analysis. In recent
year, there are continuous efforts to make high-dimensional models more
flexible and interpretable, aiming to capture more complex signals. One
particular family of such flexible and interpretable models is the
additive models where predictors are included in a model in their
functional forms. The additive models can help select predictors who
have linear or nonlinear effects and provide more accurate prediction
when nonlinear effects exist. Guo et al.~developed Bayesian hiarchical
additive models to analyze continous, categorical and survival outcomes,
and demonstrated improved prediction performance compare to the
state-of-the-art additive models. In this article, we introduce the R
package \texttt{BHAM} that implements the spike-and-slab LASSO additive
models and computationally efficient algorithms to fit these models.

The package \texttt{BHAM} provides functions for setting up and fitting
various spike-and-slab LASSO additive models, including generalized
additive models for various continuous and discrete otucoems and Cox
survival models for censored survival outcomes. These functions are
extended from previously published Bayesian Hierarchical linear models
\texttt{BhGLM}, and develop upon commonly used R functions \texttt{s} in
\texttt{mgcv} to construct additive functions. Hence, the proposed
models shares similar syntax from well-developed packages and provide
powerful feasures f these standard tools. The sytax can be easily
followed and provide user friendliness. In addition, the algorithms
implemented in \texttt{BHAM} is easily scalable, particularly suitable
for fitting high-dimensional models. In the package, we also provide a
series utility functions, for example . Hence, BHAM provides xxxx and is
helpful for xxx.

\subsection{Literature Review}

We enlist current available packages that have similar functionality,
i.e.~modeling to the best of our knowledge. To note, we don't list
packages that are unable of handling high-dimensional data, for example
the well known R package \texttt{mgcv}, and high-dimensional packages
that requires extra steps to construct the design matrix of functional
form of predictors (Such implementation can be found with grouped sparse
models, for example \texttt{SGL}.)

\cite{Scheipl2013} Summarized the software development of additive
models in high-dimensional data analysis before 2013.

\subsubsection{Generalized Additive Model}
\begin{itemize}
\item \texttt{COSSO}
\item \texttt{spikeSlabGAM}
\item \texttt{sparseGAM}
\end{itemize}

\subsubsection{Additive Cox Proportional Hazard Model}
\begin{itemize}
\item \texttt{COSSO}
\item \texttt{tfCox}
\end{itemize}

The \textbf{BHAM} package provides a scalable solution for fitting
high-dimensional generalized additive model and additive Cox model using
spike-and-slab LASSO priors or other regularized priors, including
continuous spike-and-slab priors, Student' T priors and double
exponential priors. It fits linear, logistic, poisson and Cox regression
models. The specification of the additive functions follows a popular
syntax implemented in
\href{https://cran.r-project.org/web/packages/mgcv/index.html}{\texttt{mgcv}}.
Ancillary functions are provided, including cross-validation, model
summary, and visualization.

In this article, we focus on the packages that can directly construct
additive models for high-dimensional data analysis, instead of requiring
additional step of constructing design matrix of functional form of the
variables before fitting a sparse model.

There are other methods to model survival outcome and provides
proporitonal hazards interpretation, for example \cite{Marra2021}
provides a link-based survival additive model for mixed censoring in
package \texttt{GJRM}.

\section{Models and algorithms}

\section{Features}

\clearpage
\section{Reference}



\end{document}
