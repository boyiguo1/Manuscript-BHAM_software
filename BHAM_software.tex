\documentclass[
]{jss}

\usepackage[utf8]{inputenc}

\providecommand{\tightlist}{%
  \setlength{\itemsep}{0pt}\setlength{\parskip}{0pt}}

\author{
Boyi Guo\\University of Alabama at Birmingham \And Nengjun
Yi\\University of Alabama at Birmingham
}
\title{The R Package \pkg{BHAM}: Fast and Scalable Bayeisan Hierarchical
Additive Model for High-dimensional Data}

\Plainauthor{Boyi Guo, Nengjun Yi}
\Plaintitle{The R Package BHAM: Fast and Scalable Bayeisan Hierarchical
Additive Model for High-dimensional Data}
\Shorttitle{\pkg{BHAM}: Bayeisan Hierarchical Additive Model}


\Abstract{
\textbackslash pkg\{BHAM\} is a freely avaible R pakcage that implments
Bayesian hierarchical additive models for high-dimensional clinical and
genomic data. The package includes functions that generlized additive
model, and Cox additive model with the spike-and-slab LASSO prior. These
functions implements scalable and stable algorithms to estimate
parameters. \textbackslash pkg\{BHAM\} also provides utility functions
to construct additive models in high dimensional settings, select
optimal models, summarize bi-level variable selection results, and
visualize nonlinear effects. The package can facilitate flexible
modeling of large-scale molecular data, i.e.~detecting succeptable
variables and inforing disease diagnostic and prognostic. In this
article, we describe the models, algorithms and related features
implemented in \textbackslash pkg\{BHAM\}. The package is freely
avaiable via the public GitHub repository
\url{https://github.com/boyiguo1/BHAM}.
}

\Keywords{additive model, spike-and-slab LASSO, scalable}
\Plainkeywords{additive model, spike-and-slab LASSO, scalable}

%% publication information
%% \Volume{50}
%% \Issue{9}
%% \Month{June}
%% \Year{2012}
%% \Submitdate{}
%% \Acceptdate{2012-06-04}

\Address{
    Boyi Guo\\
    University of Alabama at Birmingham\\
    1665 University Blvd\\
Birmingham, AL 35294-0002 USA\\
  E-mail: \email{boyiguo1@uab.edu}\\
  URL: \url{http://boyiguo1.github.io}\\~\\
      Nengjun Yi\\
    University of Alabama at Birmingham\\
    1665 University Blvd\\
Birmingham, AL 35294-0002 USA\\
  E-mail: \email{nyi@uab.edu}\\
  
  }

% Pandoc syntax highlighting

% Pandoc citation processing


\usepackage{amsmath}

\begin{document}

\section{Introduction}

Growing interest in fitting flexible and interpretable models,
particular in high-dimensional data analysis, for the purpose of
variable selection and predictive modelling.

Characteristic of these software includes: scalability, fast
computation, and model flexibility and implementation easiness.

\subsection{Literature Review}

We enlist current available packages that have similar functionality,
i.e.~modeling to the best of our knowledge. To note, we don't list
packages that are unable of handling high-dimensional data, for example
the well known R package \texttt{mgcv}, and high-dimensional packages
that requires extra steps to construct the design matrix of functional
form of predictors (Such implementation can be found with grouped sparse
models, for example \texttt{SGL}.)

\cite{Scheipl2013} Summarized the software development of additive
models in high-dimensional data analysis before 2013.

\subsubsection{Generalized Additive Model}
\begin{itemize}
\item \texttt{COSSO}
\item \texttt{spikeSlabGAM}
\item \texttt{sparseGAM}
\end{itemize}

\subsubsection{Additive Cox Proportional Hazard Model}
\begin{itemize}
\item \texttt{COSSO}
\item \texttt{tfCox}
\end{itemize}

The \textbf{BHAM} package provides a scalable solution for fitting
high-dimensional generalized additive model and additive Cox model using
spike-and-slab LASSO priors or other regularized priors, including
continuous spike-and-slab priors, Student' T priors and double
exponential priors. It fits linear, logistic, poisson and Cox regression
models. The specification of the additive functions follows a popular
syntax implemented in
\href{https://cran.r-project.org/web/packages/mgcv/index.html}{\texttt{mgcv}}.
Ancillary functions are provided, including cross-validation, model
summary, and visualization.

In this article, we focus on the packages that can directly construct
additive models for high-dimensional data analysis, instead of requiring
additional step of constructing design matrix of functional form of the
variables before fitting a sparse model.

There are other methods to model survival outcome and provides
proporitonal hazards interpretation, for example \cite{Marra2021}
provides a link-based survival additive model for mixed censoring in
package \texttt{GJRM}.

\hypertarget{bayesian-hierarchical-additive-model}{%
\section{Bayesian Hierarchical Additive
Model}\label{bayesian-hierarchical-additive-model}}

\hypertarget{gerenalized-additive-model}{%
\subsection{Gerenalized Additive
Model}\label{gerenalized-additive-model}}

\hypertarget{cox-proportional-hazard-model}{%
\subsection{Cox Proportional Hazard
Model}\label{cox-proportional-hazard-model}}

\hypertarget{r-functions}{%
\section{R Functions}\label{r-functions}}

\hypertarget{model-fitting}{%
\subsection{Model fitting}\label{model-fitting}}

\hypertarget{high-dimension-smoothing-formula}{%
\subsubsection{High-dimension Smoothing
Formula}\label{high-dimension-smoothing-formula}}

\begin{CodeChunk}
\begin{CodeInput}
R> x <- 1:10
R> x
\end{CodeInput}
\begin{CodeOutput}
 [1]  1  2  3  4  5  6  7  8  9 10
\end{CodeOutput}
\end{CodeChunk}

\hypertarget{model-fitting-1}{%
\subsubsection{Model Fitting}\label{model-fitting-1}}

\hypertarget{covariate-adjustment}{%
\subsubsection{Covariate Adjustment}\label{covariate-adjustment}}

\hypertarget{model-summary}{%
\subsection{Model Summary}\label{model-summary}}

\hypertarget{functional-selection}{%
\subsubsection{Functional Selection}\label{functional-selection}}

\hypertarget{curve-plotting}{%
\subsubsection{Curve Plotting}\label{curve-plotting}}

\hypertarget{model-performance}{%
\subsubsection{Model Performance}\label{model-performance}}

\section[Metabolomics Data Analysis with BHAM]{Metabolomics Data
Analysis with \proglang{BHAM}}\label{sec:analysis}

\hypertarget{contious-outcome}{%
\subsection{Contious Outcome}\label{contious-outcome}}

\hypertarget{binary-outcome}{%
\subsection{Binary Outcome}\label{binary-outcome}}

\hypertarget{survival-outcome}{%
\subsection{Survival Outcome}\label{survival-outcome}}

\hypertarget{conclusion}{%
\section{Conclusion}\label{conclusion}}

This template demonstrates some of the basic LaTeX that you need to know
to create a JSS article.

\hypertarget{code-formatting}{%
\subsection{Code formatting}\label{code-formatting}}

In general, don't use Markdown, but use the more precise LaTeX commands
instead:

\begin{itemize}
\item
  \proglang{Java}
\item
  \pkg{plyr}
\end{itemize}

One exception is inline code, which can be written inside a pair of
backticks (i.e., using the Markdown syntax).

If you want to use LaTeX commands in headers, you need to provide a
\texttt{short-title} attribute. You can also provide a custom identifier
if necessary. See the header of Section \ref{r-code} for example.

\section[R code]{\proglang{R} code}\label{r-code}

Can be inserted in regular R markdown blocks.

\subsection[Features specific to rticles]{Features specific to
\pkg{rticles}}\label{features-specific-to}

\begin{itemize}
\tightlist
\item
  Adding short titles to section headers is a feature specific to
  \pkg{rticles} (implemented via a Pandoc Lua filter). This feature is
  currently not supported by Pandoc and we will update this template if
  \href{https://github.com/jgm/pandoc/issues/4409}{it is officially
  supported in the future}.
\item
  Using the \texttt{\textbackslash{}AND} syntax in the \texttt{author}
  field to add authors on a new line. This is a specific to the
  \texttt{rticles::jss\_article} format.
\end{itemize}



\end{document}
